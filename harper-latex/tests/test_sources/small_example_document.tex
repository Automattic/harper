% Document by Grant Lemons <grantlemons@aol.com>

\documentclass[12pt,letterpaper]{article}

\usepackage[letterpaper, margin=1in]{geometry}
\usepackage[utf8]{inputenc}
\usepackage[T1]{fontenc}
\usepackage{ifpdf}
\usepackage{mla}
\usepackage{blindtext}
\usepackage{charter}
\usepackage[all]{nowidow}
\usepackage{microtype}
\usepackage[skip=1em, tocskip=1.3em, parfill]{parskip}
\usepackage{fancyhdr}
\usepackage{ragged2e}

% Bibliography Setup
\usepackage{csquotes}
\usepackage[american]{babel}
\usepackage[style=mla,backend=biber]{biblatex}
\addbibresource{bibliography.bib}

% Link setup
\usepackage{hyperref}    
\usepackage[nameinlink, noabbrev, capitalize]{cleveref}
\hypersetup{
    colorlinks=true, % set true if you want colored links
    linktoc=all,     % set to all if you want both sections and subsections linked
    linkcolor=black,  % choose some color if you want links to stand out
    citecolor=black,
    urlcolor=black,
}

\begin{document}
\begin{mla}{Grant}{Lemons}{Dr. Heather Fester}{Nature and Human Values}{August 28, 2023}{What Water am \textit{I} Swimming In?}
\justifying
    \section{This is Water Response}
        In David Foster Wallace's famous 2005 commencement speech \enquote{This is Water}, he posits to a class of graduates that the value of a liberal arts education is to teach a \enquote{critical awareness} of one's self and to encourage choice in the way one views the world.
        To do this, he first begins with a parable about the ignorance of two young fish towards water.
        Through said parable, Wallace is able to convey the idea of our ignorance of our own circumstances and the way we see the world, an idea that he ties back to a liberal arts education throughout his speech by discussing what he calls our \textbf{Default Setting}.

        The first time I heard this speech was on the first day of my High School Literature class in Junior year.
        My experiences in the time since have significantly shifted the way I view the speech and Wallace's message.
        As he admits in his speech, it is frighteningly easy to fall into the pattern of thinking in which the world revolves around you, in which you operate on your Default Setting.
        In the last two years, I have experienced exactly what he described in the \enquote{day in, day out} portion of the speech, in a way that I had not the first time I heard his message.
        To me, the principal message that Wallace tried to convey in this speech was open-mindedness, or making the choice to be open-minded.
        This connects to his second parable, about the unwavering certainty of both an atheist and a religious fellow, as well as to his description of the \enquote{day in, day out} pattern of life.

        I agree with his message of open-mindedness in general, but I think it is sometimes appropriate to be prescriptive.
        For instance, I feel that SUVs and large vehicles should be taxed more highly or have some sort of negative incentive attached to them.
        This perspective is due to my efforts to view the societal impact of SUVs (road wear is exponentially proportional to axle weight to the 4th), as well as the safety and environmental concerns regarding their use.
        My perspective here clashes somewhat with what Wallace says regarding open-mindedness.
        At one point in his speech, he says: \enquote{it's not impossible that some of these people in SUV’s have been in horrible auto accidents in the past, and now find driving so terrifying that their therapist has all but ordered them to get a huge, heavy SUV so they can feel safe enough to drive,} \autocite{water}
        which, although I understand his perspective, I don't entirely agree that one's personal circumstances absolve them of the societal impact of such a choice.
    \section{Personal Values}
        My personal values are a mess of beliefs about the individual and their relation to a society.
        Usually I default to, as Wallace describes, a kind of self-centered individualist perspective.
        Personal agency and independence as values feel natural to me, but when I think logically about what kind of person I want to be, values relating more to existence in a society emerge;
        logically I value generosity, adherence to what is best for a society as a whole, and dependability.

        These conflicting values likely come from both the deep-seated values of the American South (I'm from Texas), values such as independence and self-determination are deeply ingrained in American culture (especially in the South and West).
        An example of how deeply-ingrained in Southern culture is the prevalence of Baptist and Non-denominational Christian churches. These have developed primarily in the South and reflect these same values.
        American literature, too, contains these values, especially American works such as those of Didion, Hemmingway, or McCarthy.

        On the contrary, my beliefs in the individual's relationship to society are far newer and likely emerge from my interest in infrastructure and social programs such as public transit.
        Online communities surrounding these topics largely comes from the mid-far left, and within these communities there is a strong prevalence in the ideas of the good of the many and somewhat of a persecution of selfishness (NIMBYs).
        These values could also come from my interest in many East Asian cultures, many of which focus on the society and the group more than the individual.
        Through learning about the pros and cons of this focus in other countries and seeing how they have encouraged the infrastructure projects I find interesting (Shinkansen), perhaps some of that societal perspective has become a part of my own.
        
        Considering the conflict there, perhaps it is best for me to do some introspection to reconcile these values, or perhaps pursue my desired values in the same way Haidt describes Benjamin Franklin doing.

        If I had to sum up what Wallace means by \enquote{the water [we] are swimming in,} \autocite{water} I would describe it as our assumptions about the world that may not be as universal as we might assume.
        Examples may be privilege (in multiple forms) or my Western perspective on the world.
        Regardless, I'd like to become more aware of this water in my time in this course.

    \section{Final Portfolio Addition}
        Looking at my personal values through the lenses of the concepts learned in class, a lot stands out.
        The whole individualism vs. socialism conflict of values I reflected on at the beginning of this semester reminds me of the theme of justice we learned about, specifically the conflict between the philosophies
        of Nozick and Rawles.
        Despite this new way to categorize this inner struggle, however, the struggle itself has not changed.
        I still feel this conflicting value set of liberalism and socialism, despite examining it over the course of the semester.

        Coming into this semester, I was already familiar with utilitarianism and Kantianism, but I was unfamiliar with the concept of virtue ethics.
        From the first time I read an account of it, however, I liked the way that it sounded, and I have tried to cultivate certain values over the course of the semester.
        Many values, I have found, are easily tested in the environment of living in a dorm with others.
        Temperance, cooperation, and generosity all come to the forefront when living in the tightknit community that exists on campus, and it is, for the first time, in a context outside of family.

        Other ethical frameworks, too, I found myself unfamiliar with, though I have not engaged with them to the same degree as virtue ethics.
        Confucianism and Daoism, in particular, were philosophies I had head of, but not looked into.
        Reading the Dao de Ching, I found the principals appealing, if a little hard to follow, but difficult to incorporate into daily life.
        The story of the farmer going with the Dao, however, inspired me to take a step back and remember that school and grades are not everything, and that I, like the farmer, can accept happenings and move on.
        I feel like this has been beneficial for my mental health over the course of the semester, and I can't help but wonder if it has actually improved my academic performance.

        I really liked the idea of Confucianism as presented in the online lecture, and wish we could have focused on it more in class.
        Specifically, the example in which your father steals a goat really made me question some of my belief in other ethical frameworks, as, if I am being honest with myself, I would probably not report him.
        I liked the new perspectives that eastern philosophies brought to this course and to my ethical development as a whole.

    \printbibliography
\end{mla}
\end{document}
